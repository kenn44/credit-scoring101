\documentclass[12pt,a4paper]{amsart}
\usepackage{amsfonts}
\usepackage{amsthm}
\usepackage{amsmath}
\usepackage{amscd}
\usepackage[latin2]{inputenc}
\usepackage{t1enc}
\usepackage[mathscr]{eucal}
\usepackage{indentfirst}
\usepackage{graphicx}
\usepackage{graphics}
\usepackage{pict2e}
\usepackage{epic}
\numberwithin{equation}{section}
\usepackage[margin=2.9cm]{geometry}
\usepackage{epstopdf} 

 \def\numset#1{{\\mathbb #1}}

 

\theoremstyle{plain}
\newtheorem{Th}{Theorem}[section]
\newtheorem{Lemma}[Th]{Lemma}
\newtheorem{Cor}[Th]{Corollary}
\newtheorem{Prop}[Th]{Proposition}

 \theoremstyle{definition}
\newtheorem{Def}[Th]{Definition}
\newtheorem{Conj}[Th]{Conjecture}
\newtheorem{Rem}[Th]{Remark}
\newtheorem{?}[Th]{Problem}
\newtheorem{Ex}[Th]{Example}

\newcommand{\im}{\operatorname{im}}
\newcommand{\Hom}{{\rm{Hom}}}
\newcommand{\diam}{{\rm{diam}}}
\newcommand{\ovl}{\overline}
%\newcommand{\M}{\mathbb{M}}

\begin{document}

\title{Impact of alternative data and technics in building credit scorecard: merged scoring}


\author[\\]{Kenneth Assogba}

\address{}

\email{kennethassogba@gmail.com}














 \subjclass[2010]{Primary: 68T01. Secondary: 68T42}



 \keywords{nano credit, credit scoring model, credit risk modelling, machine learning, global financial inclusion} 



\begin{abstract}
The aim of this work is to study impact of new data and technics in building credit scorecard.
The final goal is to obtain a service that will evaluate creditworthiness of people not eligible for banking services 
and therefore improve their access to loans.
\end{abstract}

\maketitle

\section{Introduction} Credit requests, of new and existing customers, are
often evaluated by classical discrimination rules based on customers information. However, these kinds of strategies have serious limits.
Many financial service providers see people on the sidelines as risky partly because
they lack the data or other sources of information needed to identify high potential customers and their borrowing capacity.



\section{Problem}
\begin{itemize}
 \item Lack of traditional data to help build robust credit risk score system.
 \item Most credit risk systems are not designed keeping the rural segment in mind.
 \item Hard for financial institutions to build right combination of technology and talent.
 \item Hard to keep up with fast changing scientific and regulatory knowledge.
\end{itemize}


\section{Solution Approch}
\begin{itemize}
 \item Utilises remote sensing data alongside a range of traditional and alternative data points to assess a target population creditworthiness.
\end{itemize}

\section{Benefits}

\subsection{To Lending Organization}
\begin{itemize}
 \item Increase in approval rates
 \item Decrease in default rates
 \item Reduction in time it takes to process
 \item Increase automation and remove human errors
 \item Non-linear approach can increases acceptance rates and decreases default
\end{itemize}

\subsection{To Target Population}
\begin{itemize}
 \item Increased access to loan
 \item Reduction in time it takes to process
 \item Build score which can help bring them into formal lending system for larger loans in future
\end{itemize}

\section{Program}

\subsection{Completed}
This section presents the work that was done in the first week.\\
\begin{enumerate}
 \item Review CGAP's nano credit literature
 \item Do some research on KYC
 \item Read articles on credit scoring
\end{enumerate}

We have identified the data used in a classic scoring and the altinative data that we can introduce.\\
\begin{enumerate}
   \item Classic Data
   \begin{itemize}
     \item Applicant Data
     \item Credit Bureau Data
   \end{itemize}
   \item Non-Traditional Data
   \begin{itemize}
     \item Telecom Data
     \item Social network Data
     \item Mobile Data     
   \end{itemize}
\end{enumerate}

On the similar projects that we have already been able to study, the target population was generally reduced and having common characteristics.\\
For example:
\begin{itemize}
 \item Smallholder families
 \item Young graduate
 \item Young entrepreneur
 \item ...
\end{itemize}
So we wonder if we should not restrict the target population.

\subsection{Release}
This section presents the work that is going on.\\
Analyse the existing lending data and study which alternative data sources,
could improve the predictive power of lending decisions.\\
Studies of the reports and data of the World Bank and the CGAP
\begin{itemize}
 \item CGAP Smallholder Families Data Hub
 \item FY 2018 Benin Country Opinion Survey Report
 \item Benin - Global Financial Inclusion (Global Findex) Database 2014
 \item Benin - Global Financial Inclusion (Global Findex) Database 2018
\end{itemize}


\subsection{Backlog}
This section presents the work to be done.\\
\begin{itemize}
 \item Analysis of existing solutions
 \item Comparison of techniques and algorithms used in scoring.
\begin{itemize}
  \item decision trees (Breiman et al., 1984)
  \item neural networks (Mcculloch and Pitts, 1943)
  \item discriminant analysis (Fisher, 1936; Mahalanobis, 1936)
  \item logistic regression (Cox, 1970; Cox and Snell, 1989)
  \item discriminant analysis and logistic regression
  \item ...
\end{itemize}
 \item Tests
 \item Determine the tools to collect the data needed to train the model

\end{itemize}



\begin{thebibliography}{99} 

\bibitem{cgap1} CGAP (Consultative Group to Assist the Poor): \textit{CGAP and Harvesting Explore Use of Alternative Data in Credit Scores}, \begin{verbatim} cgap.org/news/cgap-and-harvesting-explore-use-alternative-data-credit-scores
\end{verbatim}
\bibitem{ben1} Farid Beninel, Waad Bouaguel, Ghazi Belmufti: (Consultative Group to Assist the Poor): \textit{Transfer Learning Using Logistic Regression in Credit Scoring}
\bibitem{bun1} Rory P. Bunker, M. Asif Naeem, Wenjun Zhang: \textit{Improving a Credit Scoring Model by Incorporating Bank Statement Derived Features}

\end{thebibliography}



\end{document}